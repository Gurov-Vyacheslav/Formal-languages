\documentclass[a4paper,12pt]{article}
\usepackage[utf8]{inputenc}
\usepackage[english,russian]{babel}
\usepackage{amsmath,amssymb}
\usepackage{tikz}
\usetikzlibrary{automata,positioning,arrows}
\usepackage{geometry}
\geometry{left=2cm,right=2cm,top=2cm,bottom=2cm}

\begin{document}
\shorthandoff{"}

\begin{center}
    {\LARGE \textbf{Отчет по лабораторной работе №2}}
    
    \vspace{0.5cm}
    
    {\large \textbf{Гуров Вячеслав ИУ9-52Б}}
    
    \vspace{0.3cm}
    
    {\large \textbf{Вариант 8}}
\end{center}

\section*{Исходное регулярное выражение}
\[
((aa|ba)(ab)^*(bb)^*)^*
\]

\section*{Детерминированный конечный автомат}

\begin{center}
\begin{tikzpicture}[shorten >=1pt, node distance=3cm, on grid, auto]
    \node[state, initial, initial text=, accepting] (q0) {$q_0$};
    \node[state] (q1) [above right=of q0] {$q_1$};
    \node[state, accepting] (q2) [right=of q1] {$q_2$};
    \node[state] (q3) [below right=of q0] {$q_3$};
    \node[state] (q4) [right=of q2] {$q_4$};
    \node[state] (q5) [below right=of q2] {$q_5$};
    \node[state, accepting] (q6) [left=of q5] {$q_6$};
    
    \path[->]
        (q0) edge node {a,b} (q1)
        
        (q1) edge node {a} (q2)
        (q1) edge node {b} (q3)
        
        (q2) edge node {a} (q4)
        (q2) edge node [left] {b} (q5)
        
        (q3) edge [loop below] node {a,b} ()
        
        (q4) edge [bend right] node [above] {a,b} (q2)
        
        (q5) edge [bend right]  node [right] {a} (q2)
        (q5) edge node [above] {b} (q6)
        
        (q6) edge [bend left] node [right] {a} (q1)
        (q6) edge [bend right] node [below] {b} (q5);
\end{tikzpicture}
\end{center}

\textbf{Описание автомата:}
\begin{itemize}
    \item Начальное состояние: $q_0$
    \item Конечные состояния: $q_0$, $q_2$, $q_6$
    \item Алфавит: $\{a, b\}$
\end{itemize}

\section*{Проверка минимальности ДКА}

\begin{center}
\begin{tabular}{|c|c|c|c|c|c|c|}
\hline
Состояние/Слово & $\varepsilon$ & a & b & bab & bb & ab \\
\hline
$(q_0)\varepsilon$ & \textbf{+} & - & - & - & - & - \\
\hline
$(q_1)a$ & - & \textbf{+} & - & - & - & - \\
\hline
$(q_2)aa$ & \textbf{+} & - & - & - & \textbf{+} & \textbf{+} \\
\hline
$(q_3)ab$ & - & - & - & - & - & - \\
\hline
$(q_4)aaa$ & - & \textbf{+} & \textbf{+} & \textbf{+} & - & - \\
\hline
$(q_5)aab$ & - & \textbf{+} & \textbf{+} & - & - & - \\
\hline
$(q_6)aabb$ & \textbf{+} & - & - & - & \textbf{+} & - \\
\hline
\end{tabular}
\end{center}
Все строки попарно различны, следовательно автомат минимален.

\section*{Малый НКА}

\begin{center}
\begin{tikzpicture}[shorten >=1pt, node distance=3cm, on grid, auto]
    \node[state, initial, initial text=, accepting] (q0) {$q_0$};
    \node[state] (q1) [above right=of q0] {$q_1$};
    \node[state, accepting] (q2) [right=of q1] {$q_2$};
   
    \node[state] (q3) [right=of q2] {$q_3$};
    \node[state] (q4) [below right=of q2] {$q_4$};
    \node[state, accepting] (q5) [left=of q4] {$q_5$};
    
    \path[->]
        (q0) edge node {a,b} (q1)
        
        (q1) edge node {a} (q2)
        
        (q2) edge node {a} (q3)
        (q2) edge node [right] {b} (q4)
        (q2) edge [bend right] node [above] {a,b} (q1)
        
        (q3) edge [bend right] node [above] {b} (q2)
        
        (q4) edge node [above] {b} (q5)
        
        (q5) edge [bend left] node [right] {a,b} (q1)
        (q5) edge [bend right] node [below] {b} (q4);
\end{tikzpicture}
\end{center}

\textbf{Описание автомата:}
\begin{itemize}
    \item Начальное состояние: $q_0$
    \item Конечные состояния: $q_0$, $q_2$, $q_5$
    \item Алфавит: $\{a, b\}$
\end{itemize}

\textbf{Таблица множеств классов эквивалентности:}
\begin{center}
\begin{tabular}{|c|c|c|c|c|c|c|}
\hline
слово/суффикс & ab & bb & bab & b & a & $\varepsilon$ \\
\hline
aa & \textbf{+} & \textbf{+}  & - & - & - & \textbf{+}  \\
\hline
aabb & - & \textbf{+} & - & - & - & \textbf{+}  \\
\hline
aaa & - & - & \textbf{+}  & \textbf{+}  & \textbf{+} & - \\
\hline
aab & - & - & - & \textbf{+}  & \textbf{+}  & - \\
\hline
a & - & - & - & - & \textbf{+} & - \\
\hline
$\varepsilon$ & - & - & - & - & - & \textbf{+}  \\
\hline
\end{tabular}
\end{center}

\newpage
\section*{Малый ПКА}

Более сложные пересечения языков приводят к автомату с большим числом состояний. Минимальный из рассмотренных автоматов был построен для следующего случая.

Исходный язык, заданный регулярным выражением, можно представить как пересечение двух языков:
\begin{enumerate}
\item Исходный язык.
\item Исходный язык, который не обязан начинаться с aa или ba (т. е. язык со словами четной длинны, в котором после четного количества букв не может идти bbab).
\end{enumerate}
Можно также пересечь 2 пункт с языком, в котором слова начинаются на ab или aa, но это даст +1 состояние.

\begin{center}
\begin{tikzpicture}[shorten >=1pt, node distance=2cm, on grid, auto]
    % Начальное состояние ПКА

     \node[state,accepting, initial, initial text=] (start) {$\&$};
    
    % Первый автомат
    \node[state] (p0) [above right=of start] {$p_0$};
    
    % Второй автомат
    \node[state, accepting] (q0) [below right=of start] {$q_0$};
    \node[state] (q1) [right=of q0] {$q_1$};
    \node[state] (q2) [below=of q1] {$q_2$};
    \node[state, accepting] (q3) [right=of q1] {$q_3$};
    
    % epsilon-переходы из начального состояния
    \path[->]
        (start) edge [bend left] node {a,b} (p0)
        (start) edge [bend right] node [below] {$\varepsilon$} (q0);
    
    % Переходы первого автомата
    \path[->]
        (p0) edge node {a} (q0)
        
    % Переходы второго автомата
    \path[->]
        (q0) edge node {a} (q1)
        (q0) edge [bend right] [above] node {b} (q2)

        (q1) edge [bend right =50] node [above] {a,b} (q0)
        
        (q2) edge node [right] {a} (q0)
        (q2) edge node {b} (q3)
        (q3) edge [bend right]  [above] node {a} (p0)
        (q3) edge [bend left] [above] node {b} (q2)
\end{tikzpicture}
\end{center}
\textbf{Описание автомата:}
\begin{itemize}
    \item Начальное состояние: $\&$
    \item Конечные состояния: $\&$, $q_0$, $q_3$
    \item Алфавит: $\{a, b\}$
\end{itemize}
\textbf{Таблица множеств классов эквивалентности:}
\begin{center}
\begin{tabular}{|c|c|c|c|c|}
\hline
слово/суффикс & $\varepsilon$ & bab & b & a \\
\hline
aaa  &- & \textbf{+} & \textbf{+} & \textbf{+}  \\
\hline
aab &- & - & \textbf{+} & \textbf{+}  \\
\hline
a & - &- & - & \textbf{+} \\
\hline
ab & - & - & - & - \\
\hline
\end{tabular}
\end{center}
Если необходим детерменизм всех состояний кроме состояния пересечения, то нужно добавить состояние ловушку:
\begin{center}
\begin{tikzpicture}[shorten >=1pt, node distance=2cm, on grid, auto]
    % Начальное состояние ПКА

     \node[state,accepting, initial, initial text=] (start) {$\&$};
    
    % Первый автомат
    \node[state] (p0) [above right=of start] {$p_0$};
    \node[state] (T) [right=4 of p0] {$T$};
    % Второй автомат
    \node[state, accepting] (q0) [below right=of start] {$q_0$};
    \node[state] (q1) [right=of q0] {$q_1$};
    \node[state] (q2) [below=of q1] {$q_2$};
    \node[state, accepting] (q3) [right=of q1] {$q_3$};
    
    % epsilon-переходы из начального состояния
    \path[->]
        (start) edge [bend left] node {a,b} (p0)
        (start) edge [bend right] node [below] {$\varepsilon$} (q0);
    
    % Переходы первого автомата
    \path[->]
        (p0) edge node {a} (q0)
        (p0) edge [below]node {b} (T)
        (T) edge [loop right] node {a,b} ()
        
    % Переходы второго автомата
    \path[->]
        (q0) edge node {a} (q1)
        (q0) edge [bend right] [above] node {b} (q2)

        (q1) edge [bend right =50] node [above] {a,b} (q0)
        
        (q2) edge node [right] {a} (q0)
        (q2) edge node {b} (q3)
        (q3) edge [bend right = 10]  [above] node {a} (p0)
        (q3) edge [bend left] [above] node {b} (q2)
\end{tikzpicture}
\end{center}

\newpage
\section*{Расширенное регулярное выражение}

Заметим, что $(aa|ba)$ эквивалентно $(a|b)a$, а $(a|b)$ можно записать как $[ab]$. Тогда расширенное регулярное выражение равно:

\texttt{\^{}([ab]a(ab)*(bb)*)*\$}

Можно пойти дальше и заменить $[ab]$ на wildcard-операцию "$.$", но тогда необходима операция предпросмотра, чтобы гарантировать, что всё слово состоит только из символов a и b:

\texttt{\^{}(?=[ab]*\$)(.a(ab)*(bb)*)*\$}

Данная запись использует положительный lookahead \texttt{(?=[ab]*\$)} для проверки, что все символы в строке принадлежат множеству $\{a,b\}$, и wildcard операцию "$.$" для сокращения записи.\\
Новое регулярное выражение можно описать автоматом:
\begin{center}
\begin{tikzpicture}[shorten >=1pt, node distance=3cm, on grid, auto]
    \node[state, initial, initial text=, accepting] (start) {$\&$};
    \node[state, accepting] (p0) [below right=of start] {$p_0$};
    \node[state] (q0) [above right=of start] {$q_0$};
    \node[state, accepting] (q1) [right=of q0] {$q_1$};
   
    \node[state] (q2) [right=of q1] {$q_2$};
    \node[state] (q3) [below right=of q1] {$q_3$};
    \node[state, accepting] (q4) [left=of q3] {$q_4$};
    
    \path[->]
        (start) edge node {"."} (q0)
        (start) edge node {a,b} (p0)

        (p0) edge [loop right] node {a,b} ()
        
        (q0) edge node {a} (q1)
        
        (q1) edge node {a} (q2)
        (q1) edge node [right] {b} (q3)
        (q1) edge [bend right] node [above] {"."} (q0)
        
        (q2) edge [bend right] node [above] {b} (q1)
        
        (q3) edge node [above] {b} (q4)
        
        (q4) edge [bend left] node [right] {"."} (q0)
        (q4) edge [bend right] node [below] {b} (q3);
\end{tikzpicture}
\end{center}
\textbf{Описание автомата:}
\begin{itemize}
    \item Начальное состояние: $\&$
    \item Конечные состояния: $\&$, $q_1$, $q_4$, $p_0$
    \item  "." - любой символ
\end{itemize}
\end{document}