\documentclass[12pt,a4paper]{article}
\usepackage[utf8]{inputenc}
\usepackage[russian]{babel}
\usepackage{amsmath}
\usepackage{amssymb}
\usepackage{mathtools}
\usepackage{tikz}
\usetikzlibrary{automata,positioning,arrows.meta}


\title{Контрольная работа №2}
\author{Гуров В.А., группа ИУ9-52б}
\date{Вариант 7}

\begin{document}

\maketitle

\section*{Задание 1}

\textbf{Исходная КС-грамматика:}
\[
S \to aSbS \mid aS \mid bab
\]
с условием: количество букв $a$ в слове в 2 раза больше $b$,\\ то есть $|w|_a = 2|w|_b$.\\

\textbf{Представление КС-грамматики в виде PDA:}
\begin{figure}[h]
\centering
\begin{tikzpicture}[
  ->,
  >=Stealth,
  node distance=26mm,
  on grid,
  auto,
  every state/.style={minimum size=10mm}
]

\node[state, initial, initial text=] (q0) {$q_0$};
\node[state] (q1) [right=of q0] {$q_1$};
\node[state] (q2) [right=of q1] {$q_2$};
\node[state, accepting] (q3) [right=of q2] {$q_3$};

\path
  (q0) edge[loop above]
       node[above] {$a, \frac{x}{Ax}$} (q0)
  (q0) edge[bend left=15]
       node {$b$} (q1)
  (q1) edge[bend left=15]
       node {$a$} (q2)
  (q2) edge[bend left=15]
       node {$b$} (q3)
  (q3) edge[bend left=25]
       node {$b, \frac{A}{\varepsilon}$} (q0);

\end{tikzpicture}
\end{figure}

$q_0$ — начальное состояние, $q_3$ — принимающее.\\

\textbf{I. Пересечение с регулярным языком:}
\[
L_{\text{reg}} = a^+ (babb)^+ a^+ bab
\]

\textbf{II. Доказательство того, что язык не является КС-языком (лемма о накачке)}

Пусть $n \geq p$ — число накачки.

Рассмотрим слово:
\[
w = a^n (babb)^n a^{4n+3} bab
\]

Подсчитаем количество букв:
\begin{align*}
|w|_a &= n + n + 4n + 3 + 1 = 6n + 4 \\
|w|_b &= 3n + 2 \\
|w|_a &= 2|w|_b \quad \checkmark
\end{align*}

Рассмотрим варианты накачки ($w = x_1 y_1 z y_2 x_2$), которые не нарушают структуру регулярного выражения:

\begin{enumerate}
    \item Если $y_1 z y_2$ находится в блоке $a^n$: при $i > 1$ количество букв $a$ увеличивается, а количество $b$ остаётся тем же. Условие $|w|_a = 2|w|_b$ нарушается.
    
    \item Если $y_1 y_2 = (babb)^k$ ($k \geq 1$): при $i = 0$ имеем
    \[
|w|_a - 2|w|_b = 6n + 4 - k - 2(3n + 2 - 3k) = 5k \neq 0
\]

    Баланс нарушается.
    
    \item Если $y_1 z y_2$ находится в блоке $a^{4n+3}$: аналогично пункту 1.
    
    \item Если $y_1 z y_2$ находится в блоке $a^n (babb)^n$ с $y_1 = a^k$, $y_2 = (babb)^l$ ($k, l \geq 1$): для сохранения баланса при $i > 1$ требуется $k = 5l$. Однако при $i = 0$ степень первого блока из $a$ равна $n - 5l$, а степень блока из $babb$ равна $n - l$, что невозможно по структуре слова (количество букв $a$ должно быть не менее количества блоков $babb$).
    
    \item Если $y_1 z y_2$ находится в блоке $(babb)^n a^{4n+3}$ с $y_1 = (babb)^k$, $y_2 = a^l$ ($k, l \geq 1$): при $i > 1$ количество блоков $babb$ становится больше количества предшествующих букв $a$, что противоречит структуре.
\end{enumerate}

Слово не накачивается, следовательно, язык не является КС-языком.

\newpage

\section*{Задание 2}

\textbf{Язык:}
\[
L = \{ c^i a^n c^* b^k c^j \mid k = n \text{ или } (i + j > 1 \text{ и } i < j) \}
\]

\textbf{Доказательство того, что язык не является DCFL:}

Пусть $n \geq p$ — число накачки.

Рассмотрим два слова:
\begin{align*}
x &= c^n a^n c^n\\
y &= c\\
z &= c b^n\\
w &= xy \\
w' &= xz
\end{align*}

\textbf{I. Попытка накачать $x$}

Разберём варианты расположения $t_1 z t_2$ (где $x = m_1 t_1 z t_2 m_2$):

\begin{enumerate}
    \item Если $t_1 z t_2$ находится в первом блоке $c^n$ ($t_1 t_2 = c^k$, $k \geq 1$): при $i = 2$ количество $c$ в начале становится $n + k$, что больше количества $c$ в конце ($n$). Ни одно из условий языка не выполняется для слова $w$.
    
    \item Если $t_1 z t_2$ находится во втором блоке $c^n$ ($t_1 t_2= c^k$, $k \geq 1$): при $i = 0$ количество $c$ во втором блоке становится $n + 1 - k \leq n$, что меньше или равно количеству $c$ в начале. Ни одно из условий не выполняется для слова $w$.
    
    \item Если $t_1 z t_2$ находится в блоке $a^n$ или включает буквы $a$: при $i = 0$ количество букв $a$ уменьшается и становится меньше $n$ (количества букв $b$). Условие $k = n$ не выполняется для слова $w'$.
\end{enumerate}

Таким образом, $x$ не накачивается.

\textbf{II. Разбиение $x$, $y$, $z$ на подслова:}

Пусть $x = x_1 x_2 x_3$, $y = y_1 y_2 y_3$, $z = z_1 z_2 z_3$ с условиями $|x_2 x_3| \leq p$ и $|x_2| > 0$.

Так как $|x_2 x_3| \leq p$, то $x_2$ находится во втором блоке $c^n$: $x_2 = c^k$ ($1 \leq k \leq p$).

При этом $y_2 = c$ или $y_2 = \varepsilon$.

При $i = 0$ количество $c$ во втором блоке уменьшается: $n + 1 - k \leq n$, что меньше количества $c$ в начале. Ни одно из условий языка не выполняется для слова $w$.

Следовательно, язык не является DCFL.

\newpage

\section*{Задание 3}

\textbf{Атрибутная грамматика:}
\[
\begin{aligned}
S &\to QSQ \quad ; \\
S &\to bb \quad ; \\
Q &\to QQ \quad ; \quad Q_1.attr \leq Q_2.attr, \, Q_0.attr := Q_1.attr \\
Q &\to aAa \quad ; \quad Q.attr := A.attr + 2 \\
A &\to BB \quad ; \quad A.attr := B_1.attr + B_2.attr \\
A &\to AA \quad ; \quad A_0.attr := A_1.attr + A_2.attr \\
B &\to b \quad ; \quad B.attr := 1
\end{aligned}
\]

\textbf{Язык, порождаемый грамматикой:}

Слово в языке представляет собой:
\[
S \to QSQ \to \cdots \to Q^n S Q^k \to (aAa)^n S (aAa)^k \to \cdots
\]
\[
\to (a(bb)^{m_i}a)^n bb (a(bb)^{m_i}a)^k,
\]
где $n, k \geq 0$ и $m_i \geq 1$.

\textbf{Анализ атрибутов:} методом пристального всматривания можно заметить, что нетерминал $Q$ может порождаться двумя правилами — первым (без условия) и третьим (с условием). Если все $Q$ по обе стороны от центрального, кроме первого, порождены третьим правилом, то минимальное требование к слову состоит в том, что в первом $Q$ количество букв $b$ не превышает количество букв $b$ в остальных $Q$ на данной стороне (для правой стороны первый $Q$ считается самым левым).

\textbf{Пересечение с регулярным языком:}

Для упрощения пересечём язык с регулярным языком:
\[
L_{\text{reg}} = (abba)bb(a(bb)^+a)^3
\]

Получится язык, состоящий из слов, образованных следующим образом:
\[
S \to QSQ \to (по\ 3\ правилу) \to QSQQQ \to (по\ 2\ и\ 4\ правилам) \to
\]
\[
(aAa)bb(aAa)^3 \to \cdots \to (abba)bb(a(bb)^{m_i}a)^3,
\]
где $m_i \geq 1$.

\textbf{Доказательство того, что язык не является КС-языком (лемма о накачке):}

Пусть $n \geq p$ — число накачки.

Рассмотрим слово:
\[
w = (abba)bb(a(bb)^na)(a(bb)^na)(a(bb)^na)
\]

Рассмотрим варианты накачки ($w = x_1 y_1 z y_2 x_2$), которые не нарушают структуру регулярного выражения. Ситуация аналогична анализу языка $\{ a^{k_1} b^{k_2} c^{k_3} \mid k_1 > k_2 \text{ и } k_1 > k_3 \}$:

\begin{enumerate}
    \item Если $y_1 y_2$ полностью находится в первом блоке $(bb)^n$ ($y_1 y_2 = (bb)^k$, $k \geq 1$): при $i = 2$ количество букв $b$ в первом блоке будет больше, чем во втором и третьем блоках. Слово не входит в язык.
    
    \item Если $y_1 y_2$ полностью находится во втором или третьем блоке $(bb)^n$ ($y_1 y_2 = (bb)^k$, $k \geq 1$): при $i = 0$ количество букв $b$ в первом блоке будет больше, чем во втором или третьем блоке соответственно. Слово не входит в язык.
    
    \item Если $y_1$ находится в первом блоке $(bb)^n$, а $y_2$ — во втором блоке $(bb)^n$ ($y_1 = (bb)^k$, $y_2 = (bb)^l$, $k, l \geq 1$): при $i = 2$ количество букв $b$ в первом блоке будет больше, чем в третьем. Слово не входит в язык.
    
    \item Если $y_1$ находится во втором блоке $(bb)^n$, а $y_2$ — в третьем блоке $(bb)^n$ ($y_1 = (bb)^k$, $y_2 = (bb)^l$, $k, l \geq 1$): при $i = 0$ количество букв $b$ в первом блоке будет больше, чем во втором и третьем блоках. Слово не входит в язык.
\end{enumerate}

Слово не накачивается, следовательно, язык не является КС-языком.

\end{document}